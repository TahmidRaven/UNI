\section*{Abstract}
With the tremendous development of deep learning (DL) and natural language processing (NLP), great potential is offered in predicting cardiac risk. We have proposed deep learning and NLP-based cardiac risk prediction using electrocardiogram (ECG) and electronic health record (EHR) data. The study aims to apply such technologies in the prediction of cardiac risks through the analysis of electrocardiograms and electronic health record data. To this end, we would like to to propose a DL-based multimodal architecture for processing the 12-lead ECG signals in tandem with NLP techniques on comments and cardiologist reports. We go further to expand this dual approach to realize our objective of improving accuracy and reliability in cardiac risk assessment. Regarding this, our study leverages data from the MIMIC-IV-EGG module, which includes about 800,000 diagnostic ECGs from nearly 160,000 unique patients. These ECGs are sampled at 500 Hz and span 10 seconds each; they are linked to the huge MIMIC-IV Clinical Database, integrating extensive patient data on demographics, diagnoses, medications, and lab results. Our approach has been to feed raw ECG signals into a deep-learning model, predict cardiac events, and generate cardiological reports. These comments, therefore, undergo further analysis with the help of LLMs so that potential diseases can be identified and refined cardiac risk predictions can be generated. Such a novel approach helps in building a vigorous predictive model that accurately identifies cardiac risks and provides detailed cardiological reports. Thus, our model enhances efficiency in the protection of cardiac health.


\vspace{1cm}
\textbf{Keywords:} Deep-Learning, Large Language Model (LLM), Natural Language Processing, ECG, EHR.
\pagebreak
