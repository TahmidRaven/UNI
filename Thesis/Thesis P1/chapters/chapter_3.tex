%\section{Problem Statements & Objectives}

\section{Problem Statements }
A lot of research has been done so far on cardiac risk prediction with ECG and structured as well as unstructured EHR datasets. The authors tried to get a higher accuracy rate by applying various models like Self Supervised Learning(SSL) models, Long Short Term Memory(LSTM), Bidirectional Long Short Term Memory(BiLSTM), Convolutional Neural Networks (CNNs) etc. However, maximum existing cardiac risk prediction models suffer limitations like single datasets (such as either ECG or EHR data) or small datasets. Some prediction models might fit well on these datasets and provide impressive results but these small datasets could have an impact on the model's adaptability. Therefore, these current models often struggle to provide a complete clinical picture of a patient's cardiovascular health. Moreover, some existing models fall short when it comes to understanding and analysing unstructured clinical data which have insightful information such as patients’ written notes and documents from doctors.
\vspace{0.5cm}


In this paper, we intend to fill these research gaps. Our study contributes to these aspects by introducing an effective and comprehensive deep learning and natural language processing based model which helps to predict cardiovascular diseases better than previous models. To do so, we integrated ECG and EHR data and applied our model to them. We employed the technique of Natural Language Processing to extract a significant amount of information from cardiologist reports which are available in our huge MIMIC-IV Clinical Database. At the same time, we analyzed the ECG signals using deep learning techniques. This model is developed to account for both the structured and unstructured data to facilitate a comprehensive and reliable development of a cardiac risk prediction model.
\vspace{1.5cm}

\section{Research Objectives }
By integrating deep learning with NLP, this research will be directed at a robust multimodal model to accurately forecast cardiac risk by analyzing 12-lead ECG signals and associated electronic health record data. Precisely, our study’s objectives are to:

i. Design and implement a DL architecture so that it can process raw ECG signals in order to predict possible cardiac events. 

ii. Use NLP on cardiologists' reports and comments and on the general trends of the EHR data to further develop and improve cardiac risk assessments.

iii. Merge the MIMIC-IV-ECG module with Clinical Database MIMIC-IV into one complete dataset, which would be more robust for training and validation.

iv. Performance and accuracy analysis of the proposed model in identifying cardiac risk and formulating detailed, actionable cardiological reports.


Meeting these objectives will ensure the research comes up with a powerful automated tool that will enhance early detection of cardiovascular diseases to support the better management of the patients for improved outcomes.
